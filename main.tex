\documentclass[10pt]{article}
\usepackage{NotesTeXV3,lipsum}
\usepackage[spanish, es-tabla]{babel}
\usepackage{cancel} 
\usepackage{graphicx}
\usepackage{xr}
\usepackage{subfiles}
\usepackage{pdfpages}
\graphicspath{{\subfix{images/}}}
%\usepackage{showframe}

\newcommand{\rref}[1]{(\ref{#1})}
\newcommand{\Figref}[1]{Figura~ \ref{#1}}
\newcommand{\Tabref}[1]{Tabla~ \ref{#1}}
\newcommand{\Eqref}[1]{Ecuación~ (\ref{#1})}


\begin{document}
	\title{{Mecánica Clásica}\\{\normalsize{\itshape Notas a un nivel intermedio}}}
	\author{Elmer Hernán Barquero Chaves}
	\affiliation{
	Estudiante de bachillerato en física por la Universidad de Costa Rica\\
	%\href{}{Website}\\
	%\href{}{Inspire-HEP}\\
	%\href{}{LinkedIn}\\
	\href{https://github.com/HernanBC}{GitHub}\\
	Estas notas son un documento en proceso por lo que es posible que contengan errores, los cuales agradecería informaran al correo, de preferencia con un asunto: `` Notas de mecánica''  o algún similar. Para realizar la corrección lo más pronto posible.

	Capítulos 2 y en adelante continuan en desarrollo, estos se colocaron para liberar los problemas resueltos. A pesar de esto, en algunos de ellos la solución no se ha podido completar, se encuentra incompleta o, en casos muy puntuales, parcialmente incorrecta.
	}
	\emailAdd{elmer.barquero@ucr.ac.cr}
	\maketitle
	\newpage
	\pagestyle{fancynotes}


	%%%%%%%%%%%%%%%%%%%%%%%%%%%%%%%
	%	Intro de las notas
	%%%%%%%%%%%%%%%%%%%%%%%%%%%%%%%

	\subfile{chapters/cap0/cap0.tex}

	%%%%%%%%%%%%%%%%%%%%%%%%%%%%%%%
	%	Cap1: Mecanica de Newton
	%%%%%%%%%%%%%%%%%%%%%%%%%%%%%%%
	\subfile{chapters/cap1/cap1.tex}

	%%%%%%%%%%%%%%%%%%%%%%%%%%%%%%%
	%	Cap2: Mecanica de Lagrange
	%%%%%%%%%%%%%%%%%%%%%%%%%%%%%%%
	\subfile{chapters/cap2/cap2.tex}

	%%%%%%%%%%%%%%%%%%%%%%%%%%%%%%%
	%	Cap3: Mecanica de Hamilton
	%%%%%%%%%%%%%%%%%%%%%%%%%%%%%%%
	\subfile{chapters/cap3/cap3.tex}

	%%%%%%%%%%%%%%%%%%%%%%%%%%%%%%%%%%%%
	%	Cap4: Transformaciones Canónicas
	%%%%%%%%%%%%%%%%%%%%%%%%%%%%%%%%%%%%
	\subfile{chapters/cap4/cap4.tex}
	
	%%%%%%%%%%%%%%%%%%%%%%%%%%%%%%%%%%%%
	%	Cap5: Mecanica del cuerpo rígido
	%%%%%%%%%%%%%%%%%%%%%%%%%%%%%%%%%%%%
	\subfile{chapters/cap5/cap5.tex}

	%%%%%%%%%%%%%%%%%%%%%%%%%%%%%%%%%%%%
	%	Cap6: Oscilaciones
	%%%%%%%%%%%%%%%%%%%%%%%%%%%%%%%%%%%%
	\subfile{chapters/cap6/cap6.tex}

	%%%%%%%%%%%%%%%%%%%%%%%%%%%%%%%%%%%%
	%	Cap7: Gravitación
	%%%%%%%%%%%%%%%%%%%%%%%%%%%%%%%%%%%%
	\subfile{chapters/cap7/cap7.tex}


	%%%%%%%%%%%%%%%%%%%%%%%%%%%%%%%%%%%%
	%	Cap8: Campos de fuerza centrales
	%%%%%%%%%%%%%%%%%%%%%%%%%%%%%%%%%%%%
	\subfile{chapters/cap8/cap8.tex}

	%%%%%%%%%%%%%%%%%%%%%%%%%%%%%%%%%%%%
	%	Cap9: Colisiones y dispersión
	%%%%%%%%%%%%%%%%%%%%%%%%%%%%%%%%%%%%
	\subfile{chapters/cap9/cap9.tex}


	%%%%%%%%%%%%%%%%%%%%%%%%%%%%%%%%%%%%
	%	Cap10: Medios Continuos
	%%%%%%%%%%%%%%%%%%%%%%%%%%%%%%%%%%%%
	\subfile{chapters/cap10/cap10.tex}



	%%%%%%%%%%%%%%%%%%%%%%%%%%%%%%%
	% bibliografía

	% Se usó ese código para generar nota en el índice y que aparezcan todos documentos que forman el archivo .bib de forma automatica sin previa cita en el text
	%%%%%%%%%%%%%%%%%%%%%%%%%%%%%%%
	\phantomsection

	\addcontentsline{toc}{section}{Referencias}
	\bibliographystyle{unsrtnat} %puede cambiarse, de ser necesario
	\bibliography{biblio.bib}
	\cite{*} %quitarlo cuando realizan las citas correspondientes

\end{document}